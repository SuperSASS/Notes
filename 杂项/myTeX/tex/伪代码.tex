\documentclass[a4paper]{article}
\usepackage[margin=1in]{geometry}
\usepackage[ruled,linesnumbered,boxed]{algorithm2e}
\usepackage{ctex}
\usepackage{amsmath}
\begin{document}
\begin{algorithm}
    \SetKwFunction{putpixel}{putpixel}
    \SetKwFunction{round}{round}
    \caption{Warshell 算法}
    \KwIn{集合元素个数 $n$,关系矩阵 $R$}
    \KwOut{传递闭包关系矩阵 $T$}

    \BlankLine

    $T \leftarrow R$\;
    \For{$k \leftarrow 1$ \KwTo $n$}
    {
        \For{$i \leftarrow 1$ \KwTo $n$}
        {
            \For{$j \leftarrow 1$ \KwTo $n$}
            {
                \If{$T_{i,k} = 1 \land T_{k,j} = 1$}
                {
                    $T_{i,j} \leftarrow 1$
                }
            }
        }
    }
\end{algorithm}

\begin{algorithm}
    \SetKwFunction{putpixel}{putpixel}
    \SetKwFunction{round}{round}
    \SetKwFunction{swap}{swap}
    \caption{中点画线法}
    \KwIn{直线$L$两端点坐标$P_0(x_0,y_0),P_1(x_1,y_1)$, 画线颜色$color$}

    \BlankLine

    \If{$x_0 > x_1$}{\swap($P_0, P_1$)\;}
    $a\leftarrow y_0-y_1, b\leftarrow x_1-x_0, c\leftarrow x_0y_1-x_1y_0$\;
    $P_i\leftarrow P_0$\;

    \uIf{$k\in[0,1]$}
    {
        $d\leftarrow 2a+b$\;
        $\Delta P_L \leftarrow (1,1), \Delta d_L \leftarrow 2(a+b)$\;
        $\Delta P_G \leftarrow (1,0), \Delta d_G \leftarrow 2a$\;
    }
    \uElseIf{$k\in(1,+\infty)$}
    {
        $d\leftarrow a+2b$\;
        $\Delta P_L \leftarrow (0,1), \Delta d_L \leftarrow 2b$\;
        $\Delta P_G \leftarrow (1,1), \Delta d_G \leftarrow 2(a+b)$\;
    }
    \uElseIf{$k\in[-1,0)$}
    {
    $d\leftarrow 2a-b$\;
    $\Delta P_L \leftarrow (1,0), \Delta d_L \leftarrow 2a$\;
    $\Delta P_G \leftarrow (1,-1), \Delta d_G \leftarrow 2(a-b)$\;
    }
    \Else
    {
        $d\leftarrow a-2b$\;
        $\Delta P_L \leftarrow (1,-1), \Delta d_L \leftarrow 2(a-b)$\;
        $\Delta P_G \leftarrow (0,-1), \Delta d_G \leftarrow -2b$\;
    }
    \While{$P \ne P_1$}
    {
        \putpixel($P$, color)\;
        \eIf{d < 0}{$P\leftarrow P + \Delta P_L, d\leftarrow d+\Delta d_L$\;}{$P\leftarrow P + \Delta P_G, d\leftarrow d+\Delta d_G$\;}
    }
    \putpixel($P_1$, color)\;
\end{algorithm}

\begin{algorithm}
    \SetKwFunction{putpixel}{putpixel}
    \SetKwFunction{round}{round}
    \SetKwFunction{swap}{swap}
    \caption{Bresenham 画线法}
    \KwIn{直线$L$两端点坐标$P_0(x_0,y_0),P_1(x_1,y_1)$, 画线颜色$color$}

    \BlankLine

    \If{$x_0 > x_1$}{\swap($P_0, P_1$)\;}
    $\Delta x = x_1-x_0, \Delta y = y_1-y_0$\;
    $P_i\leftarrow P_0$\;

    \uIf{$k\in[0,1]$}
    {
        $d\leftarrow 2\Delta y-\Delta x$\;
        $\Delta P_L \leftarrow (1,0), \Delta d_L \leftarrow 2\Delta y$\;
        $\Delta P_G \leftarrow (1,1), \Delta d_G \leftarrow 2(\Delta y - \Delta x)$\;
    }
    \uElseIf{$k\in(1,+\infty)$}
    {
        $d\leftarrow 2\Delta x-\Delta y$\;
        $\Delta P_L \leftarrow (0,1), \Delta d_L \leftarrow 2\Delta x$\;
        $\Delta P_G \leftarrow (1,1), \Delta d_G \leftarrow 2(\Delta x - \Delta y)$\;
    }
    \uElseIf{$k\in[-1,0)$}
    {
    $d\leftarrow -2\Delta y-\Delta x$\;
    $\Delta P_L \leftarrow (1,0), \Delta d_L \leftarrow -2\Delta y$\;
    $\Delta P_G \leftarrow (1,-1), \Delta d_G \leftarrow -2(\Delta y + \Delta x)$\;
    }
    \Else
    {
        $d\leftarrow 2\Delta x+\Delta y$\;
        $\Delta P_L \leftarrow (0,-1), \Delta d_L \leftarrow 2\Delta x$\;
        $\Delta P_G \leftarrow (1,-1), \Delta d_G \leftarrow 2(\Delta x + \Delta y)$\;
    }
    \While{$P \ne P_1$}
    {
        \putpixel($P$, color)\;
        \eIf{d < 0}{$P\leftarrow P + \Delta P_L, d\leftarrow d+\Delta d_L$\;}{$P\leftarrow P + \Delta P_G, d\leftarrow d+\Delta d_G$\;}
    }
    \putpixel($P_1$, color)\;
\end{algorithm}

\begin{algorithm}
    \SetKwFunction{putpixel}{putpixel}
    \SetKwFunction{round}{round}
    \SetKwFunction{swap}{swap}
    \SetKwFunction{DrawCirclePoints}{DrawCirclePoints}
    \caption{中点画圆法}
    \KwIn{圆$C$圆心$C(x,y)$,半径$R$, 画线颜色$color$}

    \BlankLine

    $P_1\leftarrow (0,R)$\;
    $d_1\leftarrow 1-R$\;
    $i\leftarrow 1$\;

    \While{$P_i \ne (\frac{R}{\sqrt{2}},\frac{R}{\sqrt{2}})$}
    {
        \DrawCirclePoints($P_i$, color)\;
        \eIf{d < 0}{$P_{i+1}\leftarrow (x_i+1,y_i), d_{i+1}\leftarrow d_{i}+2x_i+3$\;}{$P_{i+1}\leftarrow (x_i+1,y_i-1), d_{i+1}\leftarrow d_i+2(x_i-y_i)+5$\;}
        $i\leftarrow i+1$\;
    }
    \DrawCirclePoints($P_i$, color)\;
\end{algorithm}

\begin{algorithm}
    \SetKwFunction{putpixel}{putpixel}
    \SetKwFunction{round}{round}
    \SetKwFunction{swap}{swap}
    \SetKwFunction{DrawCirclePoints}{DrawCirclePoints}
    \caption{Bresenham画圆法}
    \KwIn{圆$C$圆心$C(x,y)$,半径$R$, 画线颜色$color$}

    \BlankLine

    $P_1\leftarrow (0,R)$\;
    $d_1\leftarrow 2(1-R)$\;
    $i\leftarrow 1$\;

    \While{$P_i \ne (\frac{R}{\sqrt{2}},\frac{R}{\sqrt{2}})$}
    {
        \DrawCirclePoints($P_i$, color)\;
        \eIf{$d < 0 \And 2(d+y)-1<=0$}{$P_{i+1}\leftarrow (x_i+1,y_i), d_{i+1}\leftarrow d_{i}+2x_i+3$\;}{$P_{i+1}\leftarrow (x_i+1,y_i-1), d_{i+1}\leftarrow d_i+2(x_i-y_i + 3)$\;}
        $i\leftarrow i+1$\;
    }
    \DrawCirclePoints($P_i$, color)\;
\end{algorithm}

\begin{algorithm}
    \SetKwFunction{putpixel}{putpixel}
    \SetKwFunction{round}{round}
    \SetKwFunction{swap}{swap}
    \SetKwFunction{DrawLine}{DrawLine}
    \caption{多边形逼近画圆法}
    \KwIn{圆$C$圆心$C(x,y)$,半径$R$, 画线颜色$color$}

    \BlankLine

    $n\leftarrow 3\sqrt{R}$\;
    $\theta\leftarrow \frac{2\pi}{n}$\;
    $P_1\leftarrow (0,R)$\;
    $i\leftarrow 2$\;

    \While{$i<=n$}
    {
        $P_{i} = (x_{i-1}\cos\theta - y_{i-1}\sin\theta, y_{i-1}\cos\theta + x_{i-1}\sin\theta)$\;
        \DrawLine($P_{i-1}$,$P_i$, color)\;
        $i\leftarrow i+1$\;
    }
    \DrawLine($P_{n}$,$P_1$, color)\;
\end{algorithm}

\begin{algorithm}
    \SetKwFunction{putpixel}{putpixel}
    \SetKwFunction{round}{round}
    \SetKwFunction{swap}{swap}
    \SetKwFunction{isInDrawRegion}{DrawArisInDrawRegioncInRegionPoints}
    \SetKwFunction{DrawArcInRegionPoints}{DrawArcInRegionPoints}
    \SetKwFunction{GetRegion}{GetRegion}
    \SetKwFunction{Fill}{Fill}
    \SetKwFunction{DrawCircleOfArc}{DrawCircleOfArc}
    \SetKwData{state}{state}
    \caption{任意角度画弧法}
    \KwIn{圆弧$\overset{\frown}{SE}$所属圆的圆心$C(x_c,y_c)$,圆弧起点$S(x_s,y_s)$, 圆弧终点方位点$E_{pos}(x_{epos},y_{epos}), $画线颜色$color$}
    \SetKwProg{func}{Function}{:}{end}

    \BlankLine

    \tcp{求得平移到原点的圆其对应圆心、起点、终点方位点和实际终点}
    $C'\leftarrow(0,0), S'\leftarrow(x_s-x_c,y_s-y_c)$\;
    $E_{pos}'\leftarrow(x_{epos}-x_c,y_{epos}-y_c), E'\leftarrow C'E_{pos}'\cap \odot C'$\;

    \tcp{确定起点和终点所属区域或界点}
    $region_s = \GetRegion{S},region_e = \GetRegion{E}$\;

    \tcp{划分完全绘制区域、部分绘制区域和非绘制区域}
    \Fill(\state,0)\;
    \uIf{$region_s = K_i \And region_e = K_j$}
    {
        \For{$R = R_i\to R_{j-1}$}{$\state[R] \leftarrow 1$\;}
    }
    \uElseIf{$region_s = K_i \And region_e = R_j$}
    {
        \For{$R = R_i\to R_{j-1}$}{$\state[R] = 1$\;}
        $MN\leftarrow(\min(x_{K_j},x_E),\min(y_{K_j},y_{E})), MX\leftarrow(\max(x_{K_j},x_E),\max(y_{K_j},y_{E}))$\;
        $Rect\leftarrow(MN,MX), \state[region_e]\leftarrow-1$\;
    }
    \uElseIf{$region_s = R_i \And region_e = K_j$}
    {
        \For{$R = R_{i+1}\to R_{j}$}{$\state[R] = 1$\;}
        $MN\leftarrow(\min(x_{K_{i+1}},x_S),\min(y_{K_{i+1}},y_{S})), MX\leftarrow(\max(x_{K_{i+1}},x_S),\max(y_{K_{i+1}},y_{S}))$\;
        $Rect\leftarrow(MN,MX), \state[region_s]\leftarrow-1$\;
    }
    \Else
    {
        \For{$R = R_{i+1}\to R_{j-1}$}{$\state[R] = 1$\;}
        \eIf{$S,E$ 属于同一象限 $\And S$ 在 $E$顺时针侧}{$MN\leftarrow(\min(x_S,x_E),\min(y_S,y_E)), MX\leftarrow(\max(x_S,x_E),\max(y_S,y_E))$\;$Rect\leftarrow(MN,MX)$\;}
        {
            $MN_1\leftarrow(\min(x_{K_j},x_E),\min(y_{K_j},y_{E})), MX_1\leftarrow(\max(x_{K_j},x_E),\max(y_{K_j},y_{E}))$\;
            $MN_2\leftarrow(\min(x_{K_{i+1}},x_S),\min(y_{K_{i+1}},y_{S})), MX_2\leftarrow(\max(x_{K_{i+1}},x_S),\max(y_{K_{i+1}},y_{S}))$\;
            $Rect\leftarrow(MN_1,MX_2)\cup(MN_2,MX_2)$\;
        }
        $\state[region_s]\leftarrow-1, \state[region_e]\leftarrow-1$\;
    }

    \tcp{八方对称遍历圆像素点,每次生成的像素点需要判断是否在绘制区域内}
    $\DrawCircleOfArc(C, R, color, \state, Rect)$\;

\end{algorithm}

\end{document}